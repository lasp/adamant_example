\input{common_packages.tex}

\begin{document}

\title{\textbf{Pico Example} \\
\large\textit{Assembly Design Document}}
\date{}
\maketitle

\section{Description}
\input{build/tex/pico_example_description.tex}

\section{Design}

\subsection{At a Glance}
\input{build/tex/pico_example_stats.tex}

\subsection{Components}
All of the components in the Pico Example assembly are reusable components that can be found within the Adamant repository (in \textit{src/components/}) with a few exceptions:

\vspace{5mm} %5mm vertical space
\begin{spaceditemize}
  \item \textbf{\texttt{\url{Adc_Data_Collector}}} - This component collects internal telemetry from the Raspberry Pi Pico including system voltage and temperature.
  \item \textbf{\texttt{\url{Counter}}} - This is an extremely simple component that produces an incrementing counter as telemetry.
  \item \textbf{\texttt{\url{Fault_Producer}}} - This is an component that can be used to inject a fault into the system via command.
  \item \textbf{\texttt{\url{Oscillator}}} - This is a simple component that produces two sinusoidal outputs into telemetry. 
\end{spaceditemize}
\vspace{5mm} %5mm vertical space

These specific component are located in the Example repository in \textit{src/components/}. A full list of the components included in the assembly can be seen below. Note that components marked as active execute on their own task within the Ada runtime.

\input{build/tex/pico_example_components.tex}

\subsection{Views}

This section shows the example assembly visually as a set of \textit{views}. Each shows a specific set of components and connections (while not showing other components and connections) in order to highlight a particular function of the assembly. Components that are bold are \textit{active}, meaning they have an Ada task assigned to them. Connections are labeled with the type that is passed along them. A dotted line indicates that the connection is asynchronous, meaning the data is put onto a queue for later processing. A solid line indicates that the connection is synchronous, meaning processing of that data occurs right when the data is passed along the connector.

\input{build/tex/pico_example_views.tex}

\subsection{Task Priorities}
\input{build/tex/pico_example_priorities.tex}

\subsection{Commands}

\input{build/tex/pico_example_commands.tex}

\subsection{Parameters}

\input{build/tex/pico_example_parameters.tex}

\subsection{Events}

\input{build/tex/pico_example_events.tex}

\subsection{Data Products}

\input{build/tex/pico_example_data_products.tex}

\subsection{Packets}

\input{build/tex/pico_example_packets.tex}

\subsection{Faults}

\input{build/tex/pico_example_faults.tex}

\section{Appendix}
\subsection{Connections}
\input{build/tex/pico_example_connections.tex}

\subsection{Packed Types}

\input{build/tex/pico_example_types.tex}

\subsection{Enumerations}

\input{build/tex/pico_example_enums.tex}

\end{document}
